\documentclass[a4paper, 11pt]{scrartcl}
%% PDF SETUP
\usepackage[pdftex, bookmarks, colorlinks, breaklinks,
pdfusetitle,plainpages=false]{hyperref}
\hypersetup{linkcolor=blue,pdfauthor={Zentrum für Technikkultur Landau},citecolor=blue,filecolor=black,urlcolor=blue,plainpages=false} 
\usepackage[utf8]{inputenc}
\usepackage[T1]{fontenc}
\usepackage[ngerman]{babel}
\usepackage{booktabs}
\usepackage{tabularx}
\usepackage{amssymb}
\usepackage{eurosym}

\usepackage{ccicons}
\usepackage{url}

\setlength{\parindent}{0pt} % Keine Einrückung nach Absatz

\usepackage[left=2cm,right=2cm,top=2cm,bottom=2cm]{geometry}% blendet Seitenränder ein

% Optima as a sans serif font.
\renewcommand*\sfdefault{uop}
\usepackage[protrusion=true,expansion=true]{microtype}
% Recalculate page setup based on new font.
%\KOMAoptions{DIV=last}
\pagestyle{empty}

\renewcommand*\thesection{\S~\arabic{section}}
\KOMAoptions{toc=flat}

\begin{document}

%suppress the label:
\def\LayoutTextField#1#2{% label, field
  #2%
}

\vspace{-6cm}
\title{\vspace{-2.0cm}Antrag auf Mitgliedschaft}
\subtitle{im Verein "`Zentrum für Technikkultur Landau e.\,V."'}
\author{}
\date{}

\maketitle
\thispagestyle{empty}
\vspace{-2cm}

Der Mitgliedsantrag muss ausgefüllt und unterschrieben an folgende Adresse gesendet werden:\\

\textbf{Zentrum für Technikkultur Landau e. V.\\
Peter Scheydt\\
Weißquartierstraße 31\\
76829 Landau\\
}

oder eingescannt per Mail an \href{mailto:vorstand@ztl.space}{vorstand@ztl.space} mit dem Betreff "`Beitritt"'.\\

\renewcommand{\arraystretch}{1.8}
\begin{Form}
\begin{tabularx}{\linewidth}{lX}\toprule
	Organisation:	& \TextField[width=14cm, bordercolor=1 1 1, backgroundcolor=0.98 0.98 0.98]{namea} \\ \midrule
	Name: 			& \TextField[width=14cm, bordercolor=1 1 1, backgroundcolor=0.98 0.98 0.98]{nameb} \\ \midrule
	Vorname: 		& \TextField[width=14cm, bordercolor=1 1 1, backgroundcolor=0.98 0.98 0.98]{namec} \\ \midrule
	Geburtsdatum: 	& \TextField[width=14cm, bordercolor=1 1 1, backgroundcolor=0.98 0.98 0.98]{named} \\ \midrule
	Straße: 		& \TextField[width=14cm, bordercolor=1 1 1, backgroundcolor=0.98 0.98 0.98]{namee} \\ \midrule
	PLZ, Ort: 		& \TextField[width=14cm, bordercolor=1 1 1, backgroundcolor=0.98 0.98 0.98]{namef} \\ \midrule
	E-Mail:			& \TextField[width=14cm, bordercolor=1 1 1, backgroundcolor=0.98 0.98 0.98]{nameg} \\ 
	\bottomrule
\end{tabularx}

\vspace{0.5cm}

{\tiny Bei Organisationen ist unter Vorname und Nachname eine Vertretungsberechtigte Person einzutragen. Das Feld Geburtsdatum kann entfallen. \\
Bei Familien sind alle Namen und Geburtsdaten und ggf. E-Mail-Adressen anzugeben.}

\vspace{0.5cm}

\begin{tabularx}{\linewidth}{lX}
	Art der Mitgliedschaft : 	& 	\CheckBox[name=ord, width=0.6cm, height=0.6cm, bordercolor=0 0 0]{} 
									Ordentliche Mitgliedschaft \newline (20\euro/Monat)\\
								& 	\CheckBox[name=fam, width=0.6cm, height=0.6cm, bordercolor=0 0 0]{} 
									Familienmitgliedschaft {\tiny 2 Erwachsene und max. 3 Kinder unter 18 Jahre}  \newline (30\euro/Monat)\\
								& 	\CheckBox[name=red, width=0.6cm, height=0.6cm, bordercolor=0 0 0]{} 
									Mitgliedschaft mit reduziertem Beitrag {\tiny Schüler, Studenten und Auszubildende}  \newline (10\euro/Monat)\\
								& 	\CheckBox[name=mec, width=0.6cm, height=0.6cm, bordercolor=0 0 0]{} 
									Doppelmitgliedschaft MEC {\tiny bei bestehender Mitgliedschaft im MEC Landau e. V.} \newline (10\euro/Monat)\\
								& 	\CheckBox[name=k14, width=0.6cm, height=0.6cm, bordercolor=0 0 0]{} 
									Doppelmitgliedschaft Funkamateure {\tiny bei bestehender Mitgliedschaft im DARC e. V. Ortsverband Landau - K14} \newline (10\euro/Monat)\\
								& 	\CheckBox[name=foe, width=0.6cm, height=0.6cm, bordercolor=0 0 0]{} 
									Fördermitgliedschaft {\tiny ohne Stimmrecht in der Mitgliederversammlung} 
									\newline Betrag: \TextField[width=3cm, bordercolor=1 1 1, backgroundcolor=0.98 0.98 0.98]{betrag}\euro/Monat\\\\
\end{tabularx}
\end{Form}
Fördermitglieder mit einem Beitrag ab 50\euro/Monat werden auf unserer Website aufgeführt. \\
Bei einem Beitrag ab 100\euro/Monat wird zusätzlich ein Schild an unserer Sponsorenwand installiert.

\vspace{0.5cm}

\textbf{Wir weisen gemäß Art. 13 Datenschutz-Grundverordnung darauf hin, dass zum Zweck der Mitgliederverwaltung und -betreuung folgende Daten der Mitglieder in automatisierten Dateien gespeichert, verarbeitet und genutzt werden: Namen, Anschriften, Geburtsdaten und E-Mail-Adressen. Nach der Beendigung der Mitgliedschaft werden die Daten für weitere 2 Jahre gespeichert.\\
Die Rechtsgrundlage für die Verarbeitung der personenbezognen Daten ist die Vereinsmitgliedschaft.\\
Es besteht das Recht auf Auskunft, Berichtigung, Löschung und Einschränkung bzw. Widerspruch der Verarbeitung der personenbezogenen Daten sowie ein Beschwerderecht bei der zuständigen Aufsichtsbehörde.}\\

Ich bin mit der Erhebung, Verarbeitung und Nutzung folgender personenbezogener Daten durch den Verein zur Mitgliederverwaltung im Wege der elektronischen Datenverarbeitung einverstanden: Name, Anschrift, Geburtsdatum, E-Mail-Adresse. Mir ist bekannt, dass dem Aufnahmeantrag ohne dieses Einverständnis nicht stattgegeben werden kann.\\

Hiermit beantrage ich die Aufnahme in den Verein "`Zentrum für Technikkultur Landau e. V."'. Für meine Mitgliedschaft gilt die aktuelle Satzung des Vereins wie sie unter \url{https://ztl.space/die-satzung/} veröffentlicht wurde.\\

\vspace{50pt}
\TextField[width=5cm, bordercolor=1 1 1, backgroundcolor=0.98 0.98 0.98]{nameh}\\
\noindent\rule{5cm}{.4pt}\hfill\rule{5cm}{.4pt}\par
\noindent Datum, Ort \hspace{9.8cm} Unterschrift
\vspace{50pt}

Die Mitgliedschaft beginnt nach positivem Aufnahmebescheid mit dem Eingang des ersten Mitgliedsbeitrags. Diese kann jederzeit mit einer formlosen Kündigung an vorstand@ztl.space beendet werden.

\vspace{50pt}
\begin{center}
\rule{15cm}{.4pt}\\ 
1. Vorstand: Peter Scheydt - Weißquartierstraße 31, 76829 Landau in der Pfalz \\
www.ztl.space E-Mail: vorstand@ztl.space \\
Bankverbindung: VR Bank Südpfalz - IBAN: DE81 5486 2500 0001 7636 79 - BIC: GENO DE 61 SUW

\end{center}

\end{document}
